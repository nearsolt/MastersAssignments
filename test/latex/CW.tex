\documentclass[a4paper,14pt]{extreport}
\usepackage{gost}
\usepackage{comment}
%\setcounter{page}{2}							
\usepackage{hyperref}							
%\hypersetup{hidelinks}							 
\usepackage{verbatim}							 
\usepackage{pdfpages}							 
\usepackage{cmap}
\newcommand{\RNumb}[1]{\uppercase\expandafter{\romannumeral #1\relax}}
\usepackage{xcolor}
\usepackage{listings}
\lstset{ 
	backgroundcolor=\color{white}, 
	basicstyle=\footnotesize,      
	breakatwhitespace=false,       
	breaklines=true,  
	escapeinside={\%*}{*)},        
	extendedchars=\true,            
	firstnumber=1,                 
	keepspaces=true,                   
	language=Octave,               
	morekeywords={*,...},          
	numbers=left,                  
	numbersep=15pt,
	numberstyle=\tiny,  
	stepnumber=1
}
\usepackage[title,titletoc]{appendix}
\newcommand{\empline}{\mbox{}\newline} % пустая строка
\newcommand{\append}[1]{ 
	\clearpage
	\stepcounter{chapter}
	\begin{center}
		\chaptertitlename~\Asbuk{chapter}
	\end{center}
	\begin{center}{#1}\end{center}
	\empline
	\addcontentsline{toc}{chapter}{\Asbuk{chapter}\hspace{0.6em}~#1}}

\begin{document}
\includepdf[pages={1}]{titul_cw.pdf}
\tableofcontents
\intro{

    В математике и ее приложениях постоянно приходится иметь дело с приближенными представлениями функций. Одним из инструментов для задач интерполяции в теории приближения являются сплайны. К достоинствам можно отнести их прекрасные аппроксимационные свойства, универсальность и простоту реализации. \par В 1946 году Исаак Шенберг впервые употребил термин "сплайн"\, в качестве обозначения класса полиномиальных сплайнов.\par До 1960 годов сплайны были в основном инструментом теоретических исследований, они часто появлялись в качестве решений различных экстремальных и вариационных задач, особенно в теории приближений. \par После 1960 года с развитием вычислительной техники началось использование сплайнов в компьютерной графике и моделировании. \par Сплайны имеют многочисленные применения как в математической теории, так и в разнообразных вычислительных приложениях. В частности, сплайны двух переменных используются для задания поверхностей в различных системах компьютерного моделирования.
{\color{red}
    \par В данной работе будут рассмотрены  интерполяционные сплайны первой степени одной переменной, $S_1(x)$, дефекта 1 на сетке $\Delta$, оценки погрешности для различных классов функций, сходимость интерполяционного процесса и алгоритмы  для построения сетки с меньшим числом разбиений, чем при равномерном разбиении.
    \par Цель работы заключается в исследовании  линейной интерполяции функций одной переменной сплайнами первой степени.
    \par Для достижения поставленной цели были сформулированы следующие задачи:
    \begin{itemize}
        \item ввести основные определения, необходимые для построения интерполяционного сплайна одной переменной;
        \item рассмотреть алгоритм построения сплайнов одной переменной;
        \item оценить погрешность приближения функций с помощью сплайнов;
        \item рассмотреть сходимость интерполяционного процесса и интерполяцию с заданной точностью;
        \item рассмотреть алгоритмы для построения сетки с меньшим числом разбиений, чем при равномерном разбиении.
    \end{itemize}
}
}
\definitions{

    Пусть на отрезке $[a,b]$ задано разбиение $\Delta:a=x_0<x_1<\ldots<x_N=b$. Обозначим через $C^k=C^k[a,b]$, где $k\geqslant0, k\in\mathbb{Z}$, множество $k$ раз непрерывно дифференцируемых на $[a,b]$ функций, а через $C^{-1}[a,b]$~-- множество кусочно-непрерывных функций с точками разрыва первого рода.
    \begin{definition}
        Функция $S_{n,v}(x)$ называется сплайном степени $n$ дефекта~$v$ ($v\in\mathbb{Z}, 0\leqslant v\leqslant n+1$) с узлами на сетке $\Delta$, если выполняются следующие условия:
        \begin{equation}
            \label{df_1}
            \mbox{а) }S_{n,v}(x)=\sum_{\alpha=0}^{n}a^i_\alpha(x-x_i)^\alpha\mbox{ для } x\in[x_i,x_{i+1}],\, i=\overline{0, N-1};\hspace{21mm}
        \end{equation}
        \begin{equation*}
            \mbox{б) }S_{n,v}(x)\in C^{n-v}[a,b].\hspace{107mm}
        \end{equation*}
    \end{definition}
    Определение сплайна имеет смысл и на всей вещественной оси, если положить $a=-\infty, b=+\infty$. На каждом отрезке $[x_i,x_{i+1}]$ для сплайна, помимо формулы (\ref{df_1}), возможно следующее представление:
    \begin{equation}
        \label{df_2}
        S_{n,v}(x)=\sum_{\alpha=0}^{n}b^i_\alpha(x-x_{i+1})^\alpha, i=\overline{0, N-1}.
    \end{equation}
    При этом на полуоси $(-\infty,x_1]$ берется только формула (\ref{df_2}), а на полуоси $[x_{N-1},\infty)$  только формула (\ref{df_1}).
    \par Сплайн $S_{n,v}(x)$ имеет непрерывные производные до порядка $n-v$, а производные порядка выше $n-v$ терпят разрывы в точках $x_i, i=\overline{1, N-1}$. Для определенности будем считать, что функция $S_{n,v}^{(r)}(x), r>n-v$, непрерывна справа, т.е.
    \begin{equation*}
        S_{n,v}^{(r)}(x_i)=S_{n,v}^{(r)}(x_i+0), r=\overline{n-v+1, n},\,i=\overline{1, N-1}.
    \end{equation*}
    \begin{definition}
        Множество сплайнов, удовлетворяющих определению, обозначим  через $S_{n,v}(\Delta)$. Оно является линейным множеством (линейным пространством), так как операции сложения элементов из $S_{n,v}(\Delta)$ и умножения на действительные числа не выводят за пределы множества.
    \end{definition}
    Очевидно, что этому множеству принадлежат сплайны степени $n$ дефекта $v_1<v$, сплайны степени $n_1<n$ дефекта $v_1<v$, если будет выполняться условие $n_1-v_1\geqslant n-v$, а также многочлены степени не выше $n$.
    \begin{definition}
        Функцию $f(x)\in C[a,b]$ на $\Delta:a=x_0<x_1<\ldots<x_N=b$ будем характеризовать ее колебанием на отрезках $[x_i,x_{i+1}]$:
        \begin{equation*}
            \omega_i(f)=\max\limits_{x',x''\in [x_i,x_{i+1}]}{|\,f(x'')-f(x')\,|},
        \end{equation*}
        а также величиной
        \begin{equation*}
            \omega(f)=\max\limits_{0\leqslant i\leqslant N-1}{\omega_i(f)}.
        \end{equation*}
         \par Характеристикой функции, не зависящей от сетки $\Delta$, является модуль непрерывности 
         \begin{equation*}
               \omega(f;h)=\max\limits_{\footnotesize\begin{array}c x',x''\in [a,b] \\ |\,x''-x'\,|\leqslant h\end{array}}{|\,f(x'')-f(x')\,|},\;h\leqslant b-a.
         \end{equation*}
    \end{definition}
    \begin{theorem}[о среднем]
        Если $f(x)\in C[a,b]$ и величины $\alpha,\beta$ имеют одинаковые знаки, то выполняется следующее равенство
        \begin{equation*}
            \alpha f(a)+\beta f(b)=(\alpha+\beta)f(\xi), \mbox{где }\xi\in[a,b].
        \end{equation*}
    \end{theorem}
}
\chapter{Сплайны первой степени одной переменной}
    Сплайны первой степени $S_1(x)$ дефекта 1 на сетке $\Delta$~-- непрерывные кусочно-линейные функции. 
    \begin{definition}
        Пусть в узлах сетки $\Delta:a=x_0<x_1<\ldots<x_N=b$ заданы значения $f_i=f(x_i), i=\overline{0, N}$ некоторой функции $f(x)$ определенной на $[a,b]$. Интерполяционным сплайном называется сплайн,  $S_{1}(f;x)\equiv S_{1}(x)$, удовлетворяющий условиям:
        \begin{equation}
            \label{ch_1.1}
            S(x_i)=f_i,\,i=\overline{0, N}.
        \end{equation}
    \end{definition}
    \begin{definition}
        Пусть в узлах сетки $\Delta:a=x_0<x_1<\ldots<x_N=b$ заданы значения $f_i=f(x_i), i=\overline{0, N}$ некоторой функции $f(x)$ определенной на $[a,b]$ и $h_i=x_{i+1}-x_i$ при  $x\in[x_i,x_{i+1}]$. \par Тогда уравнение сплайна будет иметь вид
        \begin{equation}
            \label{ch_1.2}
            S_1(x) = f_i\,\dfrac{x_{i+1}-x}{h_i}+f_{i+1}\,\dfrac{x-x_{i}}{h_i}\Leftrightarrow S_1(x) = f_i+\dfrac{x-x_{i}}{h_i}\,(f_{i+1}-f_i).
        \end{equation}
    \end{definition}
    Вычисление сплайна удобнее проводить в следующем порядке: сначала находим $u_{i}=\dfrac{f_{i+1}-f_{i}}{h_{i}}$, затем $S_{1}(x)=f_{i}+(x-x_{i}) u_{i}$.
    \par\vspace{3mm} Качество приближения функций, в том числе и интерполяции, характеризуется остаточным членом. В данном случае это $R(x)=S_{1}(x)-f(x)$. Оценка остаточного члена зависит от того, какими дифференциальными свойствами обладает интерполируемая функция $f(x)$. \par Пусть $W_p^k\,[a,b]$~-- множество функций, имеющих на $[a,b]$ абсолютно непрерывную производную $(k-1)$ порядка и производную $k$ порядка на пространстве $L_p\,[a,b]$. \par Рассмотрим некоторые теоремы, позволяющие сделать оценку погрешности.
    \begin{theorem}
        Если сплайн первой степени $S_{1}(x)$ интерполирует функцию $f(x)$ на сетке $\Delta$, то справедливы оценки
        \begin{equation}
            \label{ch_1.3}
            \left\|R^{(r)}(x)\right\|_{\infty}=\left\|S_{1}^{(r)}(x)-f^{(r)}(x)\right\|_{\infty} \leqslant R_{r}, \quad r=0,1,
        \end{equation}
        где $R_{r}$ определены в таблице \hyperref[table_1]{1.1}:
        \begin{table}[h]
            \label{table_1}
            \renewcommand{\arraystretch}{2.2}
            \centering
            \begin{tabular}[c]{|c|c|c|}
                \hline ~~Класс функций~~ & $R_0$ & $R_1$ \\
                \hline  $C[a,b]$ & $\omega(f)$ & -\\
                $W_{\infty}^{1}[a, b]$ & $\dfrac{\bar{h}}{2}\,\left\|\,f'(x)\,\right\|_{\infty}$& -\\
                $C C_{\Delta}^{1}[a, b]$& $\dfrac{\bar{h}}{4}\,\omega\left(f'\right)$& $\omega\left(f'\right)$\\
                $C W_{\Delta, \infty}^{2}[a, b]$& $\dfrac{\bar{h}^{2}}{8}\,\left\|\,f''(x)\,\right\|_{\infty}$& $\dfrac{\bar{h}}{2}\,\left\|\,f''(x)\,\right\|_{\infty}$\\\hline
            \end{tabular}\vspace{3mm}
            \caption{оценки погрешности для функций, принадлежащих различным классам.}
        \end{table} 
    \end{theorem}
    \begin{proof}
        Используя для $S_1(x)$ представление (\ref{ch_1.2}), при $x\in[x_i,x_{i+1}]$ получим
        \begin{equation}
            \label{ch_1.4}
            R(x)=S(x)-f(x)=(1-t)f_i+tf_{i+1}-f(x),\quad t=\dfrac{x-x_i}{h_i}.
        \end{equation}
        \par\RNumb{1}. Пусть $f(x)\in C[a,b]$. Применим к выражению $(1-t)f_i+tf_{i+1}$ теорему о среднем, получим
        \begin{equation*}
            R(x)=S(x)-f(x)=f(\xi)-f(x),\quad \xi\in[x_i,x_{i+1}].
        \end{equation*}
        Следовательно, $|\,R(x)\,|\leqslant\omega_i(f)\leqslant\omega(f)$.
        \par\vspace{3mm}\RNumb{2}. Пусть $f(x)\in W_\infty^1[a,b]$. По формуле Тейлора имеем
        \begin{equation}
            \label{ch_1.5}
            f_i=f(x)+\int\limits_x^{x_i}f'(v)\,dv,\quad f_{i+1}=f(x)+\int\limits_x^{x_{i+1}}f'(v)\,dv.
        \end{equation}
        Подставим (\ref{ch_1.5}) в (\ref{ch_1.4}), получим
        \begin{equation}
            \label{ch_1.6}
            R(x)=-(1-t)\int\limits_{x_i}^xf'(v)\,dv +t\int\limits_x^{x_{i+1}}f'(v)\,dv.
        \end{equation}
        Оценим по модулю $R(x)$ в (\ref{ch_1.6}):
        \begin{equation}
            \label{ch_1.7}
            |\,R(x)\,|\leqslant(1-t)\int\limits_{x_i}^x|\,f'(v)\,|\,dv +t\int\limits_x^{x_{i+1}}|\,f'(v)\,|\,dv.
        \end{equation}
        В (\ref{ch_1.7}) к каждому из интегралов применим неравенство Гельдера 
        \begin{equation}
            \label{ch_1.8}
            |\,R(x)\,|\leqslant\left[(1-t)\int\limits_{x_i}^xdv +t\int\limits_x^{x_{i+1}}dv\right]\,\|\,f'(x)\,\|_\infty=2t(1-t)h_i\,\|\,f'(x)\,\|_\infty.
        \end{equation}
        Рассмотрим $t(1-t)$. Подставив значения $t$, получим следующую оценку
        \begin{equation*}
            t(1-t)=\dfrac{(x-x_i)(x_{i+1}-x)}{(x_{i+1}-x_i)^2}\leqslant\dfrac14\,.
        \end{equation*}
        Следовательно, $|\,R(x)\,|\leqslant\dfrac12\,\bar{h}\,\|\,f'(x)\,\|_\infty$.
        \par\vspace{3mm}\RNumb{3}. Пусть $f(x)\in CC_\Delta^1[a,b]$. Найдем оценку $R_0$. По формуле Тейлора с остаточным членом в форме Лагранжа имеем
        \begin{equation}
            \label{ch_1.9}
            f_i=f(x)-th_if'(\xi),\quad f_{i+1}=f(x)+(1-t)h_if'(\eta),\quad \xi,\eta\in[x_i,x_{i+1}].
        \end{equation}
        Подставим (\ref{ch_1.9}) в (\ref{ch_1.4}), получим
        \begin{equation}
            \label{ch_1.10}
            R(x)=t(1-t)h_i\,\left[f'(\eta)-f'(\xi)\right].
        \end{equation}
        Оценим по модулю $R(x)$ в (\ref{ch_1.10}):
        \begin{equation}
            \label{ch_1.11}
            |\,R(x)\,|\leqslant t(1-t)h_i\omega_i(f')\leqslant\dfrac14\,\bar{h}\omega(f').
        \end{equation}
        Найдем теперь оценку $R_1$. Из (\ref{ch_1.4}) следует
        \begin{equation}
            \label{ch_1.12}
            R'(x)=\dfrac{f_{i+1}-f_i}{h_i} - f'(x).
        \end{equation}
        Используя (\ref{ch_1.9}), получим
        \begin{equation}
            \label{ch_1.13}
            R'(x)=(1-t)f'(\eta)+tf'(\xi)-f'(x)
        \end{equation}
        Применим теорему о среднем, получим
        \begin{equation*}
            R'(x)=f'(\zeta)-f'(x),\quad \zeta\in[x_i,x_{i+1}].
        \end{equation*}
        Следовательно, $|\,R'(x)\,|\leqslant\omega(f')$.
        \par\vspace{3mm}\RNumb{4}. Пусть $f(x)\in CW_{\Delta,\infty}^2[a,b]$. Найдем оценку $R_0$. По формуле Тейлора имеем
        \begin{equation}
            \label{ch_1.14}
            f_{i}=f(x)-t h_{i} f'(x)+\int\limits_{x}^{x_{i}}(x_{i}-v) f''(v) dv,
        \end{equation}
        \begin{equation}
            \label{ch_1.15}
            f_{i+1}=f(x)+(1-t) h_{i} f'(x)+\int\limits_{x}^{x_{i+1}}(x_{i+1}-v) f''(v) dv .
        \end{equation}
        Из формулы (\ref{ch_1.4}) следует, что 
        \begin{equation}
            \label{ch_1.16}
            R(x)=(1-t) \int\limits_{x}^{x_{i}}(x_{i}-v) f''(v) dv+t \int\limits_{x}^{x_{i+1}}(x_{i+1}-v) f''(v) dv.
        \end{equation}
         Оценим по модулю $R(x)$ в (\ref{ch_1.16}), применив неравенство Гельдера, получим
         \begin{equation}
            \label{ch_1.17}
            |\,R(x)\,|\leqslant\dfrac12\,h_i^2t(1-t)\,\|\,f''(x)\,\|_\infty\leqslant\dfrac18\,\bar{h}^2\,\|\,f''(x)\,\|_\infty.
        \end{equation}
        Оценка $R_1$ выводится аналогичными рассуждениями из \RNumb{3} этапа из (\ref{ch_1.12}):
        \begin{equation}
            \label{ch_1.18}
            |\,R(x)\,|\leqslant\dfrac12\,(1-2t+2t^2)h_i\,\|\,f''(x)\,\|_\infty\leqslant\dfrac12\,\bar{h}\,\|\,f''(x)\,\|_\infty.
        \end{equation}
        \par Таким образом, получили оценки погрешности из таблицы \hyperref[table_1]{1.1}. Теорема доказана.
    \end{proof}
    \par\noindent\textbf{Теорема 2.}
    Если $f(x) \in C C_{\Delta}^{k}[a, b], k \geqslant 3$, то для $x \in\left[x_{i}, x_{i+1}\right]$


\begin{comment}
    \par\noindent\textbf{Теорема 2.}
Если $f(x) \in C C_{\Delta}^{k}[a, b], k \geqslant 3$, то для $x \in\left[x_{i}, x_{i+1}\right]$
\begin{equation}
    \label{5}
    S_{1}(f ; x)=f(x)+t(1-i) \sum_{\alpha=2}^{k-1} \frac{h_{i}^{\alpha}}{\alpha !} f^{(\alpha)}(x)\left[(-1)^{\alpha} t^{\alpha-1}+(1-t)^{\alpha-1}\right]+O\left(h_{i}^{k}\right),
\end{equation}\vspace{-5mm}
\begin{multline}
    \label{6}
    S_{1}^{\prime}(f ; x)=f^{\prime}(x)+\frac{1-2 t}{2} h_{i} f^{\prime \prime}(x)+\\+\sum_{\alpha=3}^{k-1} \frac{h_{i}^{\alpha-1}}{\alpha !} f^{(\alpha)}(x)\left[(-1)^{\alpha-1} t^{\alpha}+(1-t)^{\alpha}\right]+O\left(h_{i}^{k-1}\right).
\end{multline}
\par\noindent\textbf{Следствие 1.}
Если $f(x) \in C C_{\Delta}^{3}[a, b]$, то 
\begin{equation}
    \label{7}
    S_{1}^{\prime}\left(f ; x_{i}+\frac{h_{i}}{2}\right)=f^{\prime}\left(x_{i}+\frac{h_{i}}{2}\right)+O\left(h_{i}^{2}\right), \quad i=\overline{0,N-1}.
\end{equation}
\par Если учесть, что $  S_{1}^{\prime}\left(f ; x_{i}+\dfrac{h_{i}}{2}\right)=\dfrac{f_{i+1}-f_{i}}{h_{i}}$, то равенство (\ref{7}) представляет собой известную оценку приближения производной с помощью центрально-разностной аппроксимации.
\par Далее перейдем к рассмотрению вопроса сходимости интерполяционного процесса. Пусть на $[a, b]$ задана сетка $\Delta_{v}:a=x_{v, 0}<x_{v, 1}<\ldots<x_{v,N_{v}} =b$, $v=\overline{1,\infty}$, удовлетворяющая условию $\bar{h}_{\mathrm{v}} \rightarrow 0$ при $v \rightarrow \infty$, где
\[ \bar{h}_{v}=\max _{0<i<N_{v}-1} h_{v, i}, \quad h_{v, i}=x_{v, i+1}-x_{v, i}\]
\par Предположим, что $f(x) \in C[a, b]$ и что эта функция может быть вычислена в любой точке отрезка. Для каждого $\Delta_{v}$ можно построить интерполяционный сплайн $S_{1, \Delta_{v}}(x)$. Тем самым на последовательности $\left\{\Delta_{v}\right\}$ будет определен интерполяционный процесс. Говорят, что указанный интерполяционный процесс сходится в $C[a, b]$, если
\begin{equation*}
    \left\|S_{1, \Delta_{v}}(x)-f(x)\right\|_{C[a, b]} \rightarrow 0 \mbox{ при } v \rightarrow \infty
\end{equation*}
для любой функции $f(x) \in C[a, b]$.
\parАналогичное понятие сходимости вводится и для других классов функций. В случае, если $f(x)$ дифференцируема, рассматриваем также сходимость производной $S_{1, \Delta_{v}}^{\prime}(x)$ к $f^{\prime}(x)$. Если 
\begin{equation*}
    \left\|{S}_{1, \Delta_{v}}(x)-f\left(x\right)\right\|_{C[a, b]}=O\left(\bar{h}_{v}^{\gamma}\right),
\end{equation*}
то говорят, что имеет место сходимость с порядком $\gamma$.
\par На практике обычно требуется осуществить интерполяцию с некоторой заданной точностью. Эта задача решается путем выбора сетки на отрезке $[a, b]$ с учетом свойств интерполируемой функции и соответствующих оценок погрешности интерполяции. Точность интерполяции определяется в основном гладкостью функции внутри промежутков $\left[x_{i}, x_{i+1}\right]$, а оценки погрешности интерполяции дают возможность определить величину максимального шага сетки для обеспечения заданной точности. 
\par На практике использование равномерной сетки не является оптимальным вариантом. При работе с большим числом узлов интерполяции возникает необходимость хранения большого объема информации  об узлах и узловых значениях сплайна. В данной работе будет рассмотрены два алгоритма построения сетки при которых число узлов будет меньшим, чем при равномерном разбиении.


\end{comment}
\conclusions{

}
\begin{thebibliography}{}
1 Завьялов, Ю. С. Методы сплайн~-~функций / Ю. С. Завьялов, Б. И. Квасов, В. Л. Мирошниченко.~-- М.: Наука, 1980.~- 355 с.\\ 
2 Алберг, Дж. Теория сплайнов и ее приложения / Дж. Алберг, Э. Нильсон, Дж. Уолш.~-- М.: Мир, 1972.~- 319 с.\\ 
3 Тарасов, В. Н. Численные методы. Теория, алгоритмы программы / В. Н. Тарасов, Н. Ф. Бахарева.~-- Оренбург: ИПК ОГУ, 2008.~- 266 с. \\
4 Шарый, С. П. Курс вычислительных методов / С. П. Шарый.~-- Новосибирск: Новосиб. гос. ун-т., 2016.~- 525 с.\\
5 Иванов, А. П. Численные методы: учебное пособие для факультета прикладной математики процессов управления. В 2 ч. Ч. 2 / А. П. Иванов, И. В. Олемской.~-- Санкт-Петербург, 2012.~-- 80 с.\\
6 Стечкин, С. Б. Сплайны в вычислительной математике / С. Б. Стечкин, Ю. Н. Субботин.~-- М.: Наука, 1976.~- 248 с.\\
7 Роженко, А. И. Абстрактная теория сплайнов : учеб. пособие / А. И. Роженко.~-- Новосибирск: Изд. центр НГУ, 1999.~- 176 с.\\
8 Корнейчук, Н. П. Сплайны в теории приближений / Н. П. Корнейчук.~-- М.: Наука, 1984.~- 352 с.\\ 
9 Василенко, В. А. Сплайн-функции: теория, алгоритмы, программы / В. А. Василенко.~-- Новосибирск: Наука, 1983.~- 215 с.\\
10 Брудный, Ю. А. Теория сплайнов / Ю. А. Брудный, В. К. Шалашов.~-- Ярославль: ЯГУ, 1983.~- 91 с.\\
11 Завьялов, Ю. С. Сплайны в инженерной геометрии / Ю. С. Завьялов, В. А. Леус, В. А. Скороспелов.~-- М: Машиностроение, 1985.~- 224 с.\\
\end{thebibliography}

\begin{comment}
6 Задорожный, А. Г. Построение сплайнов с использованием библиотеки OpenGL: учебное пособие / А. Г. Задорожный, Д. С. Киселев.~-- Новосибирск: НГТУ, 2019.~- 88 с.\\
7 Херн, Д. Компьютерная графика и стандарт OpenGL, 3-е издание / Д. Херн, М. Паулин Бейкер.~-- М.: Издательский дом "Вильямс"\,, 2005.~- 1168~с.\\
8 Шелевицкий, И. В. Интерполяционные сплайны в задачах цифровой обработки сигналов. / И. В. Шелевицкий /  Exponenta Pro. Математика в приложениях.~-- 2003, №4.~-- С. 42-53.\\
9 Ruijters, D. GPU Prefilter for Accurate Cubic B-spline Interpolation / Daniel Ruijters, Philippe Thévenaz / The Computer Journal.~-- 2012.~-- V. 55, Iss. 1.~-- P. 15–20.\\
10 Бахвалов, Н. С. Численные методы / Н. С. Бахвалов, Н. П. Жидков, Г. М. Кобельков.~-- М.: Наука, 2003.~- 632 с.\\
11 Пригарин, С. М. Численный анализ (интерполяция, численное дифференцирование и интегрирование) : учеб. пособие / С. М. Пригарин.~-- Новосибирск: Новосиб. гос. ун-т., 2018.~- 90 с.\\
15 Малоземов, В. Н. Полиномиальные сплайны / В. Н. Малоземов, А. Б. Певный.~-- Л.: Изд-во Ленингр. университета, 1986.~- 120 с.\\
16 Землякова, И. В. Численные методы: учебное пособие / И. В. Землякова, О. Б. Садовская, А. С. Илюхина.~-- Кострома: Изд-во Костром. гос. технол. ун-та, 2011.~- 94 с.\\
20 Бердышев, В. И. Численные методы приближения функций / В. И. Бердышев, Ю. Н. Субботин.~-- Свердловск: Средне-Уральское книжное издательство, 1979.~- 120 с.\\
\end{comment}
\end{document}
